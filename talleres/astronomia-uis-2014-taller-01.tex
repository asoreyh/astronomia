\documentclass[a4paper,12pt]{article}
\usepackage[spanish]{babel}
\hyphenation{co-rres-pon-dien-te}
%\usepackage[latin1]{inputenc}
\usepackage[utf8]{inputenc}
\usepackage[T1]{fontenc}
\usepackage{graphicx}
\usepackage{amsmath}

\usepackage[pdftex,colorlinks=true, pdfstartview=FitH, linkcolor=blue, citecolor=blue, urlcolor=blue, pdfpagemode=UseOutlines, pdfauthor={H. Asorey}, pdftitle={Introducción a la Física - Guía 01} pdfkeywords={vectores}]{hyperref}
\usepackage[adobe-utopia]{mathdesign}

\hoffset -1.23cm
\textwidth 16.5cm
\voffset -2.0cm
\textheight 26.0cm

\begin{document}
\begin{center}
  {\small{Universidad Industrial de Santander - Escuela de Física}}\\
  {\bf{Astronomía para Poetas (Asorey)}}\\
  \vspace{0.4cm}
  Taller 01: Astronomía de Posición\\ 2014
\end{center}

\renewcommand{\labelenumi}{\arabic{enumi})}
\renewcommand{\labelenumii}{\arabic{enumii})}

\begin{enumerate}

  \item Encuentre la constelación que se encontraba en el cenit, a las
    $11$\,pm, el día de su nacimiento
  
  \item ¿Cuántos años deben pasar para que el corrimiento por precesión de la
    Tierra sea de 3$^\mathrm{o}$?

  \item Determinar el tiempo que le toma a una constelación en ir desde el
    Oriente, hasta el Occidente

  \item Determine el momento exacto en que ocurrieron los equinoccios del año
    2014

  \item Verifique en que intervalo de fechas el Sol se encuentra en la
    constelación de Acuario.

  \item Encuentre a que distancia angular se encuentra la estrella Polaris
    ($\alpha$UMi) del Polo Norte Celeste
  
  \item Encuentre a que distancia angular se encuentra la estrella Polaris
    Australis ($\sigma$Oct) del Polo Sur Celeste

  \item Obtenga las coordenadas azimutales, las coordenadas ecuatoriales y los
    nombres comunes de las cuatro estrellas más brillantes (identificadas como
    $\alpha, \beta, \gamma$ y $\delta$ en orden decreciente de brillo) de las
    constelaciones de Escorpión, Orión, Osa Mayor y Cruz del Sur. Para las
    coordenadas azimutales suponga un observador en la ciudad de Bucaramanga el
    30 de Octubre de 2014 a las 6pm hora local.

  \item Para un observador situado sobre la Tierra en las coordenadas
    geográficas $(\varphi,\lambda)$, parte del cielo no será visible. Encuentre
    la relación entre la latitud $\varphi$ y la declinación $\delta$ que
    determinan si una estrella es visible o no.

  \item Para un observador situado en la Ciudad de Bucaramanga $(+7^\mathrm{o}
    07',-73^\mathrm{o} 10')$, diga si las siguientes estrellas serán visibles o
    no:\\

    \begin{center}
    \begin{tabular}{lrr}
      \hline
      {\bf Nombre} & {\bf Ascensión Recta} & {\bf Declinación} \\
      \hline
      Antares     & 16h29m24.5s     & -26$^\mathrm{o}25'55.6''$ \\ 
      $\pi$Oct    & 15h01m51.2s     & -83$^\mathrm{o}13'39.2''$ \\
      Fomalhaut  & 22h57m39.9s     & -29$^\mathrm{o}37'22.7''$ \\
      $\sigma$Oct & 21h08m47s       & -88$^\mathrm{o}57'23.1''$ \\
      Sirius      & 6h45m44s        & -16$^\mathrm{o}43'15.2''$ \\
      $\delta$Oct & 14h26m54.7s     & -83$^\mathrm{o}40'03.5''$ \\
      \hline
    \end{tabular}
    \end{center}

  \item Una estrella tiene una ascensión recta de $77^\mathrm{o}36'$ y un
    ángulo horario de $35^\mathrm{o}10'$ para cierto observador. ¿Cuál es el
    tiempo sideral local del observador?

  \item Un observador en Bucaramanga, $\lambda=73^\mathrm{o} 8'$ Oeste, mide para una estrella un ángulo horario $H=45^\mathrm{o}30'$. Si en el instante de la observación, el tiempo sideral local es 17h30m, ¿cuál es la ascensión recta $\alpha$ de la estrella?
    
\end{enumerate}
\end{document}
